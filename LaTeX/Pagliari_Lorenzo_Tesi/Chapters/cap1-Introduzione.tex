\chapter{Introduzione}
\label{chap:Introduzione}

Il lavoro che presentiamo si colloca nell'ambito dello studio del consumo energetico, nell'ambito della comunicazione tra dispositivi mobili e nell'ambito della diffusione di informazioni. Abbiamo affrontato queste tre tematiche nell'analisi di particolari scenari in cui può accadere che, sotto particolari condizioni o eventi, le comuni reti di comunicazione, quali reti telefoniche fisse, mobili e rete Internet, possano non essere disponibili per un certo lasso di tempo. Non stiamo parlando di disservizi dovuti a interventi di ordinaria manutenzione, ma di mancato servizio causato da eventi esterni che possono intaccare queste infrastrutture. Le più comuni cause di questo tipo sono i fenomeni atmosferici. Fenomeni meteorologici di forte intensità e potenza, come nubifragi, tempeste, trombe d'aria, forti nevicate, alluvioni e così molte altri, possono danneggiare le infrastrutture di comunicazione causando la mancanza dei servizi. Infatti è comune che dopo forti nevicate, grandi piogge o alluvioni, vengano a mancare i servizi telefonici e non si riesca a comunicare. In tali situazioni si resta isolati anche per giorni senza la possibilità di avere notizie di alcun tipo, anche dagli organi competenti addetti a intervenire. Situazioni di questo genere non sono le sole, ci può essere anche la mancanza volontaria dei servizi telefonici. Questo tipo di comportamento è stato e viene tuttora usato contro cortei di protesta, col tentativo di togliere alle persone il principale mezzo di organizzazione della protesta \cite{wemakehistory2014-articolo}, \cite{wemakehistory2014-fattoq}, \cite{wemakehistory2014-lastampa}. Il nostro studio ha affrontato questo problema cercando di proporre una soluzione che possa essere utilizzata in queste situazioni di emergenza. Come sottolinea l'indagine ISTAT di dicembre 2014\cite{istat2014}, la tecnologia ormai è qualcosa di molto comune tra persone, quindi abbiamo cercato di analizzare in che modo fosse possibile sfruttarla per diffondere le informazioni. Quello che abbiamo pensato è stato di concentrarci su quei dispositivi che normalmente le persone hanno con se, come telefoni cellulari o tablet, nell'ipotesi che ne siano in possesso anche in queste situazioni di emergenza. In questo caso si può immaginare, da un punto di vista di rete di comunicazione, di avere un insieme di nodi che possono vedere e comunicare solo con i nodi a loro vicini. Questo si traduce in una rete Peer-to-Peer, ovvero una rete senza \textit{overlay}, in cui non vi sono strutture gerarchiche e i nodi, chiamati \textit{peer}, comunicano tra loro solo in modalità punto-punto (da cui \textit{Peer-to-Peer}), senza alcuna struttura di tipo Client-Server. Presenteremo anche una tecnologia di comunicazione estremamente diffusa su tutti i dispositivi mobili da molti anni: il Bluetooth. La nostra idea principale in fase di progettazione della soluzione, è stata quella di simulare il comportamento umano in queste situazioni: fare il passa parola. Questo tipo di comportamento rappresenta un ottimo modo di diffondere informazioni tra le persone se non vi sono altri mezzi ed è uguale al gossip. Quando le persone vengono a conoscenza di uno scoop, un'informazione nuova, cominciano a comunicarlo ad altre persone, che a loro volta lo diranno ad altre ancora e così via. Questo è esattamente il comportamento definito come "gossip". Questo concetto vale anche per i social network, dove una nuova notizia viene \textit{tweettata} o condivisa molte volte con lo scopo di diffonderla il più possibile. Per questa ragione abbiamo scelto di studiare e utilizzare gli algoritmi di gossip, chiamati anche algoritmi epidemici, come principio di diffusione. In origine questi algoritmi vennero creati per modellare il comportamento epidemico delle malattie, ma poi si scoprì che si adattano molto bene al rappresentare anche il comportamento del gossip. Presenteremo alcuni dei più comuni algoritmi di gossip e il loro utilizzo nel progettare la nostra soluzione. 
%Parleremo anche di software e più precisamente di simulatori di reti e protocolli. Presenteremo quelli più diffusi presenti in letteratura, le loro caratteristiche e le loro funzionalità. Mostreremo anche quale simulatore è stato utilizzato per il lavoro e come esso è stato impostato.

Dalla scelta dei dispositivi da utilizzare, quali ad esempio gli smartphone, è sorto anche un altro aspetto da dover analizzare: quello del consumo energetico. E' risaputo che l'autonomia è uno dei punti deboli di questi dispositivi, quindi abbiamo lavorato per gestire questo aspetto. Presenteremo lo studio di fattibilità energetica fatto sulla tecnologia Bluetooth presa in considerazione, poi discuteremo come questo problema del consumo energetico è stato gestito.

La soluzione che proponiamo in questo lavoro quindi è un sistema che cerca di diffondere informazioni in maniera broadcast, cercando di replicare il passa parola fatto dalle persone, sfruttando come mezzo canale di trasmissione la tecnologia Bluetooth. La diffusione dell'informazione è guidata dai principi degli algoritmi di gossip ed il sistema gestisce dinamicamente il carico di lavoro da eseguire, in base alle condizioni esterne e interne del dispositivo. Viene quindi presentato un simulatore di rete che implementa gli algoritmi proposti e permette di sperimentare le varie soluzioni al fine trovare quella più adatta alle situazioni in esame. I risultati ottenuti vengono quindi presentati e discussi in dettaglio.

\section{Struttura del documento}
In questo documento presentiamo il lavoro svolto, organizzato con la seguente struttura:
\begin{itemize}
	\item \textit{Cap.2-Stato dell'arte}: in questo capitolo presentiamo lo stato dell'arte concernente le tecnologie, i modelli studiati, gli strumenti utilizzati e gli studi su cui questo lavoro si basa. Discuteremo degli studi sul risparmio energetico, di come questo aspetto sia sempre di maggiore importanza e delle ricerche fatte fin ora. Presenteremo le caratteristiche della più recente versione della tecnologia Bluetooth, elencando i principali aspetti studiati e utilizzati in questo lavoro. Illustreremo quell'insieme di modelli di rete adatti alla rappresentazione di reti Peer-to-Peer. Successivamente introdurremo il principio di Gossip e presenteremo i principali algoritmi epidemici presenti in letteratura ed infine discuteremo un insieme di strumenti di simulazione e cercheremo di approfondire gli aspetti di due di essi.
	\item \textit{Cap.3-Introduzione del problema}: in questo capitolo, discuteremo come il problema è stato affrontato e come abbiamo impostato il processo di studio. Presenteremo l'analisi di fattibilità relativa al consumo energetico generato dal Bluetooth, che sta alla base della lavoro svolto.
	\item \textit{Cap.4-Progettazione logica}: in questo capitolo presentiamo la fase di progettazione logica. Illustreremo come abbiamo strutturato il lavoro e le scelte implementative fatte per i vari moduli del sistema.
	\item \textit{Cap.5-Architettura del sistema}: in questo capitolo presentiamo l'architettura del sistema, in un ottica implementativa. Illustreremo come è stato implementato tutto ciò che è stato presentato nel Capitolo 4.
	\item \textit{Cap.6-Simulazioni e valutazione risultati}: in questo capitolo discuteremo inizialmente la parte di realizzazione, simulazione e raccolta dati per poi presentare, valutare e discutere i dati raccolti.
	\item \textit{Cap.7-Conclusioni e sviluppi futuri}: in questo capitolo presenteremo le nostre conclusioni in merito al lavoro svolto e discuteremo degli aspetti da studiare e delle possibili direzioni future da poter intraprendere per proseguire e migliorare questo lavoro.
\end{itemize}