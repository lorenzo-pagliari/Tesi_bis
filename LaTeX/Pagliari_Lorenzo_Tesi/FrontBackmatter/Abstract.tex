%*******************************************************
% Abstract
%*******************************************************
%\renewcommand{\abstractname}{Abstract}
\addcontentsline{toc}{chapter}{\abstractname}

\pdfbookmark[1]{Sommario}{Sommario}
\begingroup
\let\clearpage\relax
\let\cleardoublepage\relax
\let\cleardoublepage\relax

\chapter*{Sommario}
Questo lavoro si posiziona nell'ambito del risparmio energetico e nell'ambito delle applicazioni per dispositivi mobili. Abbiamo studiato una possibile soluzione in grado di trasmettere informazioni attraverso dispositivi mobili, quando si verificano scenari nei quali le comuni reti di telecomunicazione non sono disponibili. Scenari di questo tipo possono essere causati facilmente da forti e avverse condizioni meteorologiche. Grossi nubifragi, forti nevicate o altri eventi di questo genere, possono facilmente danneggiare la struttura della rete di telecomunicazioni, causando l'inattività delle stesse. La nostra idea è stata di progettare un sistema in grado di sfruttare i comuni dispositivi elettronici più diffusi e, tramite le tecnologie a disposizione, trovare un modo alternativo per diffondere le informazioni tra le persone. Lavorando con dispositivi mobili, e' stato necessario considerare le problematiche collegate al consumo energetico. Uno degli obiettivi del sistema che presenteremo in questo elaborato, è quello di regolare il carico di lavoro che si vuole assegnare al dispositivo, in modo da trovare sempre un compromesso tra prestazioni e autonomia residua della batteria.
In questo elaborato presenteremo la soluzione da noi proposta, illustrando i principi degli algoritmi utilizzati per guidare la diffusione dell'informazione, i modelli usati per descrivere la rete di dispositivi mobili e la tecnologia di trasmissione utilizzata.
\endgroup